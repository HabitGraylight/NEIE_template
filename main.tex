
\documentclass{document}
\usepackage[backend=biber,style=gb7714-2015]{biblatex} % 使用国标样式
\addbibresource{assets/bib/references.bib} % 加载参考文献数据库


\begin{document}
\makecovercn
  {毕业设计题目}
  {学生姓名}
  {学号}
  {指导教师}
  {助理教师}
  {专业}
  {国家卓越工程师学院}
  {2025}

\makecoveren
  {XXX Computational Model}
  {WANG Jianhua}
  {Prof. YANG XX}
  {Associate Prof. LI XX} 
  {Intelligent Connected Vehicle}
  {Intelligent Manufacturing}
  {chonghqing university}
  {2025} 

\begin{cquabstract}
    摘要是论文(设计)内容不加注释和评论的简短陈述,应以第三人称陈述。它应具有独立性和自含性,即不阅读论文(设计)的全文,就能获得必要的信息。
摘要一般应说明研究工作的目的和意义、研究思想和方法、研究过程、研究结果和最终结论等。摘要中一般不用图、表、化学结构式、计算机程序,不用非公知公用的符号、术语和非法定的计量单位。
摘要页置于英文题名页后。 
中文摘要一般为300~400字,用宋体小四号。 
关键词是指从论文(设计)的标题、摘要、正文中抽取的对表达论文(设计)主题起关键作用,且具有检索意义的词语。关键词应体现论文(设计)特色,具有语义性,在论文(设计)中有明确的出处。一般每篇论文(设计)应选取3-5个词作为关键词,以显著的字符另起一行,排在同种语言摘要的下方,尽量用《汉语主题词表》或各专业主题词表提供的规范词。
\end{cquabstract}
  
\begin{cqukeywords}
    XXXX;XXXX;XXXX
\end{cqukeywords}

  % 英文摘要
\begin{cquabstracten}
  The abstract is a concise statement of the content of the thesis (design) without annotations or comments. It should be compact and refined.
\end{cquabstracten}

% 英文关键词
\begin{cqukeywordsen}
  XXXX; XXXX; XXXX
\end{cqukeywordsen}
  
\begin{cqucontents}
%   \tableofcontents
\end{cqucontents}




\pagestyle{essaystyle}
\pagenumbering{arabic} % 页码为阿拉伯数字


\section{研究背景}
研究背景部分需要详细说明研究的动机和相关领域的现状。

\subsection{国内外研究现状}
国内外研究现状部分需要总结已有的研究成果,并指出研究的不足。

\subsubsection{国内研究现状}
国内研究现状部分需要重点分析国内的研究进展。

\subsubsection{国外研究现状}
国外研究现状部分需要重点分析国外的研究进展。

\section{研究意义}
研究意义部分需要说明本研究的创新点和实际应用价值。
摘要是论文(设计)内容不加注释和评论的简短陈述,应以第三人称陈述。它应具有独立性和自含性,即不阅读论文(设计)的全文,就能获得必要的信息。
摘要一般应说明研究工作的目的和意义、研究思想和方法、研究过程、研究结果和最终结论等。摘要中一般不用图、表、化学结构式、计算机程序,不用非公知公用的符号、术语和非法定的计量单位。\cite{YXXY20250307001}
摘要页置于英文题名页后。 
中文摘要一般为300~400字,用宋体小四号。 
关键词是指从论文(设计)的标题、摘要、正文中抽取的对表达论文(设计)主题起关键作用,且具有检索意义的词语。关键词应体现论文(设计)特色,具有语义性,在论文(设计)中有明确的出处。一般每篇论文(设计)应选取3-5个词作为关键词,以显著的字符另起一行,排在同种语言摘要的下方,尽量用《汉语主题词表》或各专业主题词表提供的规范词。\cite{10.1145/3583780.3615277}

\section{理论分析}
\subsection{模型建立}
\subsubsection{模型dada}
根据《荷载规范》,本工程上屋面活荷载取值为 $q_w = 1.5 \, \text{kN/m}^2$。

\begin{table}[htbp]
    \centering
    \caption{左风作用下简截计算}
    \begin{tabular}{cccccc}
      \toprule
      层次 & $Z(m)$ & $\mu$ & $\beta$ & $q_w (\text{kN/m}^2)$ & $W (\text{kN})$ \\
      \midrule
      1 & 4.5 & … & … & … & … \\
      2 & … & … & … & … & … \\
      \bottomrule
    \end{tabular}
  \end{table}
\subsection{实验设计}
摘要是论文(设计)内容不加注释和评论的简短陈述,应以第三人称陈述。它应具有独立性和自含性,即不阅读论文(设计)的全文,就能获得必要的信息。
摘要一般应说明研究工作的目的和意义、研究思想和方法、研究过程、研究结果和最终结论等。摘要中一般不用图、表、化学结构式、计算机程序,不用非公知公用的符号、术语和非法定的计量单位。
摘要页置于英文题名页后。   git add

中文摘要一般为300~400字,用宋体小四号。 
关键词是指从论文(设计)的标题、摘要、正文中抽取的对表达论文(设计)主题起关键作用,且具有检索意义的词语。关键词应体现论文(设计)特色,具有语义性,在论文(设计)中有明确的出处。一般每篇论文(设计)应选取3-5个词作为关键词,以显著的字符另起一行,排在同种语言摘要的下方,尽量用《汉语主题词表》或各专业主题词表提供的规范词

% 引用文献\cite{key2023}
\clearpage
\printbibliography

\end{document}